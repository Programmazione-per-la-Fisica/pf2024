
\section*{Introduzione}

\begin{frame}{La squadra}

  \begin{itemize}
  \item Dott. Carlo Battilana, titolare modulo laboratorio A-L
  \item Dott. Fabio Ferrari, titolare modulo laboratorio M-Z
  \item Dott. Samuele Lanzi, tutor A-L
  \item Dott. Alessandro Fuschi, tutor M-Z
  \end{itemize}

\end{frame}

\begin{frame}{Perché Programmazione a Fisica?}

  \textit{Programmazione}: In informatica, la stesura di un programma in una
  forma interpretabile dal calcolatore elettronico (o direttamente o dopo una
  rielaborazione opportuna), compiuta dal programmatore sulla base delle
  specifiche di programma fornite dall'analista; la programmazione comprende la
  definizione del problema da risolvere, la preparazione del diagramma di flusso
  del programma, la sua codifica (cioè la trascrizione, in un linguaggio
  opportuno, delle varie istruzioni), la messa a punto e la preparazione di una
  documentazione a uso dell'operatore.

  {\tiny (da \url{https://www.treccani.it/vocabolario/programmazione/})}

\end{frame}

\begin{frame}{Perché Programmazione a Fisica? \insertcontinuationtext}

  \begin{itemize}[<+->]
  \item La ricerca in Fisica (sperimentale, teorica o applicata) dipende
    molto spesso dalla disponibilità di sistemi informatici avanzati
  \item L'intelligenza di questi sistemi sta in gran parte nei programmi (il
    \textit{software}) che vi vengono eseguiti
  \item Vale in generale per molte altre discipline scientifiche/tecnologiche
  \item Ma il software diventa sempre più pervasivo in molte attività umane
  \end{itemize}

\end{frame}

% sondaggio
\begin{frame}{Breve sondaggio conoscitivo}
  \begin{center}
    \vfill
    \url{https://forms.office.com/e/26pfx6RvpX}
    \vfill
    \includegraphics[height=.6\textheight]{images/sondaggio-qr.png}
    \vfill
  \end{center}
\end{frame}

\begin{frame}{Materiale di supporto}

  \begin{itemize}[<+->]

  \item Presentazioni, guide, esercizi, \ldots a partire da
    \url{https://github.com/programmazione-per-la-fisica/}
    \begin{itemize}[<.->]
    \item Materiale in parte replicato su Virtuale
    \end{itemize}

  \item Questa presentazione è distribuita in due modalità:
    \begin{itemize}[<.->]
    \item un unico documento in formato \code{pdf} aggiornato incrementalmente a
      ogni lezione, disponibile a partire da\\
      {\smaller \url{https://github.com/programmazione-per-la-fisica/pf2024}}\\
      \textbf{Non} è adatto alla stampa o a prendere appunti (ci sono
      animazioni)
    \item più documenti in formato \code{pdf} separati per ogni argomento,
      caricati su Virtuale prima di ogni lezione, adatti alla stampa e a
      prendere appunti
    \end{itemize}

  \item La presentazione:
    \begin{itemize}[<.->]
    \item ha il testo in inglese, che è la lingua dell'informatica
    \item per me è un ausilio per tenere la lezione
    \item \textbf{per voi è una sintesi del contenuto del corso, non una dispensa}
    \item non può essere la vostra unica fonte per imparare a programmare
    \end{itemize}

  \end{itemize}

\end{frame}

\begin{frame}{Bibliografia e risorse consigliate}

  \begin{itemize}

  \item \textit{Learn \Cpp{}}, \url{https://learncpp.com/}: risorsa online con
    tutorial, esempi, esercizi

  \item Come referenza online: \Cpp{} reference,
    \url{https://cppreference.com/}. Piuttosto formale, ma con esempi utili

  \item
    \href{https://isocpp.github.io/CppCoreGuidelines/CppCoreGuidelines}{\textit{\Cpp{}
        Core Guidelines}}: raccolta online di linee guida per il corretto uso
    del \Cpp{}

  \item B.~Stroustrup,
    \href{https://stroustrup.com/programming.html}{\textit{Programming:
        Principles and Practice Using C++}}, $3^{rd}$ edition, Addison-Wesley

  \end{itemize}

  \begin{itemize}
  \item Evitate di consultare ulteriori risorse, soprattutto online, che spesso
    sono obsolete o di scarsa qualità
  \item Usate ChatGPT e strumenti analoghi con molta cautela
  \end{itemize}

\end{frame}

\begin{frame}{Come ci si prepara per Programmazione?}

  \begin{itemize}
  \item Se possibile, si seguono le lezioni
  \item Si ripetono in autonomia gli esempi usati a lezione e si fanno gli
    esercizi proposti (almeno all'inizio un paio d'ore la settimana)
  \item Se non si riesce a concluderla in aula, si completa in
    autonomia l'esercitazione proposta in laboratorio, possibilmente
    affrontando anche i suggerimenti per l'approfondimento
  \item Si fanno esercizi
  \item Se c'è qualcosa che non è del tutto chiaro, si chiede un
    chiarimento, anche privatamente
  \item Non scoraggiatevi. All'inizio la materia probabilmente risulta
    incomprensibile
  \end{itemize}

\end{frame}

\begin{frame}{Canali di comunicazione}

  \begin{itemize}[<+->]
  \item Virtuale

    \begin{itemize}[<.->]
    \item Usato soprattutto per annunci ufficiali relativi al calendario di
      lezioni e laboratori
    \item Strumento per consegnare il lavoro svolto in laboratorio e il progetto
      per l'esame
    \item Iscrivetevi al più presto
    \end{itemize}

  \item Ricevimento

    \begin{itemize}[<.->]
    \item In presenza o via Teams, su appuntamento
    \item Contattatemi via e-mail o chat Teams
    \end{itemize}

  \item Chat del corso?

    \begin{itemize}[<.->]
    \item Può essere uno strumento molto utile per comunicazioni
      veloci, ma gli altri anni non ha avuto successo
    \item Lasciamo l'iniziativa a voi
    \end{itemize}

  \item Commenti, domande e suggerimenti sono benvenuti in ogni forma e in ogni
    momento
    \begin{itemize}[<.->]
    \item Non aspettate la valutazione finale (che comunque è importante)
    \end{itemize}

  \end{itemize}

\end{frame}

\begin{frame}{Modalità d'esame}

  L'esame consiste in due prove:

  \begin{enumerate}

  \item Progetto riguardante l'implementazione di un programma nel linguaggio
    \Cpp{}. Il progetto è svolto in parte durante le ore di laboratorio, in
    parte in autonomia. E' raccomandato lo svolgimento in gruppo (massimo 4
    persone, dipende dalla complessità del progetto).

    Maggiori dettagli verso aprile, ma potete consultare
    \href{https://github.com/Programmazione-per-la-Fisica/progetto2023}{la
      consegna dell'anno scorso} per avere un'idea.

  \item Colloquio riguardante la discussione del progetto e domande
    teoriche e pratiche sugli argomenti svolti a lezione, spesso
    usando \href{https://godbolt.org/}{Compiler Explorer} (vedi oltre).

    Al colloquio si accede solo con una valutazione sufficiente del progetto.

  \end{enumerate}

  All'orale si porta anche il lavoro svolto durante le esercitazioni di
  laboratorio, a dimostrazione dell'effettivo svolgimento.

\end{frame}

\begin{frame}{Sistemi operativi e strumenti vari}
  \begin{itemize}[<+->]
  \item Il sistema operativo di riferimento è Ubuntu Linux
  \item Oltre a Linux, forniamo supporto per installare e configurare
    \href{https://github.com/Programmazione-per-la-Fisica/howto/blob/main/other-OSes/WSLGuide.md}{Windows}
    e
    \href{https://github.com/Programmazione-per-la-Fisica/howto/blob/main/other-OSes/macOSGuide.md}{macOS}
      \begin{itemize}[<.->]
      \item Vedi calendario prossimi appuntamenti
      \end{itemize}
  \item Per scrivere programmi si usa un \textit{\textbf{text} editor}
    \begin{itemize}[<.->]
    \item Noi raccomandiamo \href{https://code.visualstudio.com/}{\textbf{Visual
          Studio Code}}
    \end{itemize}
  \item Imparate a usare strumenti online, in particolare
    \href{https://godbolt.org/}{\textbf{Compiler Explorer}}, che
    useremo anche all'orale
  \end{itemize}

\end{frame}

\begin{frame}{Orario primo semestre}
  \begin{itemize}
  \item Lezione/esercitazione: lunedì ore 14-16
  \item Laboratorio: martedì ore 9-13 (M-Z) e 14-18 (A-L), su due
    turni ciascuno, in Aula 2 della sede in via Irnerio
    \begin{itemize}
    \item La partecipazione a Laboratorio è \textbf{obbligatoria}
    \item Indicativamente avremo 3 laboratori il primo semestre e 5 il secondo,
      secondo un calendario che annunceremo (quindi non faremo lab \textit{ogni}
      martedì, anche se dovesse apparire nell'orario)
    \item Se ce l'avete, verrete con il vostro portatile
    \end{itemize}
  \end{itemize}
\end{frame}

\begin{frame}{Prossimi appuntamenti}
  \begin{itemize}

  \item Lunedì 23/9 ore 14-16 lezione

  \item Martedì 24/9 in base ai turni dei lab, Aula 2 Irnerio
    \begin{itemize}
    \item Installazione/configurazione computer personali
    \item Fortemente raccomandato per chi non ha già un computer con Linux
    \end{itemize}

  \item Lunedì 30/9 ore 14-16 lezione

  \item Martedì 1/10 in base ai turni dei lab, Aula 2 Irnerio
    \begin{itemize}
    \item Introduzione a Linux/Unix
    \item Fortemente raccomandato per chi non ha familiarità con Linux/Unix, in
      particolare con la \textit{command line}
    \end{itemize}

  \item Lunedì 7/10 ore 14-16 lezione

  \item Martedì 8/10 in base ai turni dei lab, Aula 2 Irnerio
    \begin{itemize}
    \item (Probabile) primo laboratorio
    \item Obbligatorio
    \end{itemize}

  \end{itemize}
\end{frame}

\begin{frame}{Course outline}
  \begin{itemize}
  \item<1-> Elements of computer architecture and operating systems
  \item<2-> Introduction to Linux/Unix
  \item<3-> Why \Cpp{}
  \item<4-> Objects, types, variables
  \item<4-> Expressions
  \item<4-> Statements and structured programming
  \item<4-> Functions
  \item<4-> User-defined types and classes
  \item<4-> Generic programming and templates
  \item<4-> The Standard Library, containers, algorithms
  \item<4-> Error management
  \item<4-> Dynamic memory allocation
  \item<4-> Dynamic polymorphism (aka object-oriented programming)
  \item<5-> Elements of software engineering and supporting tools
  \end{itemize}
\end{frame}
